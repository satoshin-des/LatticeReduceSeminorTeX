\RequirePackage{plautopatch}
\documentclass[12pt,aspectratio=169,table,dvipdfmx, leqno]{beamer}

\usepackage[square,numbers]{natbib}
\usepackage{bxdpx-beamer}
\usepackage{mathtools}
\usepackage{amsmath,amsthm,amssymb,amscd}
\usepackage{ascmac}
\usepackage{physics}
\usepackage{here}
\usepackage{url}
\usepackage{bm}
\usepackage{subcaption}
\usepackage{color}
\usepackage{multirow}
\usepackage[T1]{fontenc}
\usepackage{newtxtext} % 数式以外をTXフォントで上書き
\usepackage{ulem}
\usepackage[deluxe,uplatex]{otf} % 日本語多ウェイト化
%\usepackage[linesnumbered,ruled]{algorithm2e}
\usepackage{algorithm}
\usepackage[noend]{algpseudocode}
\usepackage{tikz}

%\addbibresource{ref.bib}

%Beamerの設定
\usetheme{Boadilla}

%Beamerフォント設定
\usefonttheme{professionalfonts} % Be professional!
\renewcommand{\familydefault}{\sfdefault}  % 英文をサンセリフ体に
\renewcommand{\kanjifamilydefault}{\gtdefault}  % 日本語をゴシック体に
\renewcommand{\Comment}[1]{\quad/*~#1~*/}
\newcommand{\vol}{\text{vol}}
\newcommand{\round}[1]{\left\lfloor #1 \right\rceil}
\newcommand{\bhline}{\noalign{\hrule height 1.0pt}}
\DeclareMathOperator{\Pot}{Pot}
\DeclareFontFamily{U}{mathx}{}
\DeclareFontShape{U}{mathx}{m}{n}{<-> mathx10}{}
\DeclareSymbolFont{mathx}{U}{mathx}{m}{n}
\DeclareMathAccent{\widecheck}{0}{mathx}{"71}

\algdef{SE}[DOWHILE]{Do}{doWhile}{\algorithmicdo}[1]{\algorithmicwhile\ #1}%

% Babel (日本語の場合のみ・英語の場合は不要)
\uselanguage{japanese}
\languagepath{japanese}
\deftranslation[to=japanese]{Theorem}{定理}
\deftranslation[to=japanese]{Lemma}{補題}
\deftranslation[to=japanese]{Example}{例}
\deftranslation[to=japanese]{Examples}{例}
\deftranslation[to=japanese]{Definition}{定義}
\deftranslation[to=japanese]{Definitions}{定義}
\deftranslation[to=japanese]{Problem}{問題}
\deftranslation[to=japanese]{Solution}{解}
\deftranslation[to=japanese]{Fact}{事実}
\deftranslation[to=japanese]{Proof}{証明}
\def\proofname{証明}

\setbeamertemplate{theorems}[numbered]
\newtheorem{Proposition}[theorem]{命題}

%タイトル
\title[勉強会]{勉強会形式ゼミ資料②\\L. Ducas, L. N. Pulles and M. Stevens Towards a modern LLL implementation\cite{DPS25}}
\author[佐藤]{佐藤 新}
\date{\today}
%\institute[]{\inst{1}~立教大学}

\begin{document}
\begin{frame}
    \maketitle
\end{frame}

\begin{frame}
    本セミナーで用いられる記号など
    \begin{itemize}
        \item $\bm{A}\in M_{n, m}(\mathbb{R})$に対して,$\norm{\bm{A}}_{\max}\coloneqq \max_{i, j}\abs{a_{i, j}}$
    \end{itemize}
\end{frame}

\begin{frame}{Seysen簡約(1/3)}
\begin{itemize}
    \item $\{\bm{b}_1,\ldots, \bm{b}_n\}$: 基底
    \item $\bm{B}=(\bm{b}_1^\top, \ldots, \bm{b}_n^\top)^\top$: 基底行列
    \item $\bm{B}=\bm{RQ}~(\bm{R}:~\text{下三角行列},~\bm{Q}:~\text{直交行列})$
    \item $\bm{R}=\mqty[\bm{R}_{1, 1} & \bm{O}_{\lfloor n/2\rfloor, n-\lfloor n/2\rfloor}\\ \bm{R}_{2, 1} & \bm{R}_{2, 2}]$
\end{itemize}
\begin{definition}[Seysen簡約]
    $\{\bm{b}_1,\ldots, \bm{b}_n\}$がSeysen簡約されているとは
    \[
        \bm{R}_{1, 1},~\bm{R}_{2, 2}\text{がSeysen簡約されている}\land \norm{\bm{R}_{2, 1}\bm{R}_{1, 1}^{-1}}_{\max}\le \frac{1}{2}
    \]
\end{definition}
\end{frame}

\begin{frame}{Seysen簡約(2/3)}
    Seysen簡約は大まかには次のようなことを$\bm{B}=\bm{RQ}$なる$\bm{R}$に対して再帰的に行う.
    \begin{enumerate}
        \item $\bm{R}$を$\mqty[\bm{R}_{1, 1} & \bm{O}\\ \bm{R}_{2, 1} & \bm{R}_{2, 2}]$とブロックに分ける
        \item 
    \end{enumerate}
\end{frame}

\begin{frame}{Seysen簡約(3/3)}
\begin{algorithm}[H]
    \footnotesize
    \begin{algorithmic}[1]
        \caption{\footnotesize Seysen簡約\cite{DPS25}}
        \label{alg_seysen}
        \Require{下三角行列$\bm{R}\in M_{n}(\mathbb{R})$~s.t.~$\bm{B}=\bm{RQ}$}
        \Ensure{$\bm{UB}$が簡約基底行列となるようなunimodular行列$\bm{U}\in M_{n}(\mathbb{R})$}
        \If{$\bm{R}\in M_{1}(\mathbb{R})$}
            \State \textbf{return}~$[1]$
        \EndIf
        \State $\mqty[\bm{R}_{1, 1} & \bm{O}\\ \bm{R}_{2, 1} & \bm{R}_{2, 2}]\gets \bm{R}$ \Comment{$\bm{R}_{1, 1}\in M_{\lfloor n\rfloor}(\mathbb{R})$,$\bm{R}_{2, 1}\in M_{n-\lfloor n\rfloor, \lfloor n\rfloor}(\mathbb{R})$,$\bm{R}_{2, 2}\in M_{n-\lfloor n\rfloor, n-\lfloor n\rfloor}(\mathbb{R})$}
        \State $\bm{U}_{1, 1}\gets \texttt{seysenReduce}(\bm{R}_{1, 1})$
        \State $\bm{U}_{2, 2}\gets \texttt{seysenReduce}(\bm{R}_{2, 2})$
        \State $\bm{R}_{2, 1}\gets \bm{U}_{2, 2}\bm{R}_{2, 1}$
        \State $\bm{U}_{2, 1}\gets \round{-\bm{R}_{2, 1}\bm{R}_{1, 1}^{-1}}$
        \State $\bm{R}_{2, 1}\gets \bm{U}_{2, 1}\bm{R}_{1, 1}+\bm{R}_{2, 1}$
        \State \textbf{return}~$\mqty[\bm{U}_{1, 1} & \bm{O}\\ \bm{U}_{2, 1}\bm{U}_{1, 1} & \bm{U}_{2, 2}]$
    \end{algorithmic}
\end{algorithm}
\end{frame}

\begin{frame}[allowframebreaks]{参考文献}
\beamertemplatetextbibitems
\bibliographystyle{plain}
\typeout{}
\bibliography{ref}
\end{frame}

\end{document}
