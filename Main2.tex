\RequirePackage{plautopatch}
\documentclass[12pt,aspectratio=169,table,dvipdfmx, leqno]{beamer}

\usepackage[square,numbers]{natbib}
\usepackage{bxdpx-beamer}
\usepackage{mathtools}
\usepackage{amsmath,amsthm,amssymb,amscd}
\usepackage{ascmac}
\usepackage{physics}
\usepackage{here}
\usepackage{url}
\usepackage{bm}
\usepackage{subcaption}
\usepackage{color}
\usepackage{multirow}
\usepackage[T1]{fontenc}
\usepackage{newtxtext} % 数式以外をTXフォントで上書き
\usepackage{ulem}
\usepackage[deluxe,uplatex]{otf} % 日本語多ウェイト化
%\usepackage[linesnumbered,ruled]{algorithm2e}
\usepackage{algorithm}
\usepackage[noend]{algpseudocode}
\usepackage{tikz}

%\addbibresource{ref.bib}

%Beamerの設定
\usetheme{Boadilla}

%Beamerフォント設定
\usefonttheme{professionalfonts} % Be professional!
\renewcommand{\familydefault}{\sfdefault}  % 英文をサンセリフ体に
\renewcommand{\kanjifamilydefault}{\gtdefault}  % 日本語をゴシック体に
\renewcommand{\Comment}[1]{\quad\texttt{/*}~#1~\texttt{*/}}
\newcommand{\vol}{\text{vol}}
\newcommand{\round}[1]{\left\lfloor #1 \right\rceil}
\newcommand{\bhline}{\noalign{\hrule height 1.0pt}}
\DeclareMathOperator{\Pot}{Pot}
\DeclareFontFamily{U}{mathx}{}
\DeclareFontShape{U}{mathx}{m}{n}{<-> mathx10}{}
\DeclareSymbolFont{mathx}{U}{mathx}{m}{n}
\DeclareMathAccent{\widecheck}{0}{mathx}{"71}

\algdef{SE}[DOWHILE]{Do}{doWhile}{\algorithmicdo}[1]{\algorithmicwhile\ #1}%

% Babel (日本語の場合のみ・英語の場合は不要)
\uselanguage{japanese}
\languagepath{japanese}
\deftranslation[to=japanese]{Theorem}{定理}
\deftranslation[to=japanese]{Lemma}{補題}
\deftranslation[to=japanese]{Example}{例}
\deftranslation[to=japanese]{Examples}{例}
\deftranslation[to=japanese]{Definition}{定義}
\deftranslation[to=japanese]{Definitions}{定義}
\deftranslation[to=japanese]{Problem}{問題}
\deftranslation[to=japanese]{Solution}{解}
\deftranslation[to=japanese]{Fact}{事実}
\deftranslation[to=japanese]{Proof}{証明}
\def\proofname{証明}

\setbeamertemplate{theorems}[numbered]
\newtheorem{Proposition}[theorem]{命題}

%タイトル
\title[勉強会]{勉強会形式ゼミ資料②\\L. Ducas, L. N. Pulles and M. Stevens Towards a modern LLL implementation\cite{DPS25}}
\author[佐藤]{佐藤 新}
\date{\today}
%\institute[]{\inst{1}~立教大学}

\begin{document}
\begin{frame}
    \maketitle
\end{frame}

\begin{frame}
    本セミナーで用いられる記号など
    \begin{itemize}
        \item $\bm{A}\in M_{n, m}(\mathbb{R})$に対して,$\norm{\bm{A}}_{\max}\coloneqq \max_{i, j}\abs{a_{i, j}}$
        \item $\bm{A}\in M_{n, m}(\mathbb{R})$に対して,$\bm{A}_{[r, s]\times [t, u]}\coloneqq [a_{i, j}]_{r\le i\le s, t\le j\le u}$
    \end{itemize}
\end{frame}

\begin{frame}{\cite{DPS25}の貢献}
    \begin{itemize}
        \item BLASterの提案
        \begin{itemize}
            \item 高速かつモダンなLLLの実装
            \item \cite{NS16}の分割手法を利用
            \item サイズ簡約をSeysen簡約に置き換え
            \item Cholesky分解をQR分解に置き換え
        \end{itemize}
    \end{itemize}
\end{frame}

\begin{frame}{\cite{NS16}の分割手法}
\begin{minipage}[b]{0.45\columnwidth}
\begin{algorithm}[H]
    \footnotesize
    \begin{algorithmic}[1]
        \caption{\footnotesize Reduce\cite{NS16}}
        \label{alg_REDUCE}
        \Require{格子$L$の基底行列$\bm{B}\in M_{n}(\mathbb{Z})$,簡約パラメタ$0<\eta\le 1$}
        \Ensure{簡約された基底$\{\bm{b}_1,\ldots,\bm{b}_n\}$}
        \State $n_r\gets n$以下最大のspecific~dimension
        \State computes GSO~$B\gets[\norm{\bm{b}_i}^\star], \bm{U}\gets [\mu_{i, j}]$
        \State $\texttt{sizeReduce}(\bm{B})$
        \State $\overline{\beta}\gets 2+\lceil \log_2{(\sqrt{n}(\max_{i\le n}\norm{\bm{b}_i})^n)}\rceil$
        \For{$t=1$~to~$(n/n_r)^2\lceil \log(\overline{\beta}/\eta)\rceil$}
            \For{$k=0$~to~$n-n_r$}
                \State $B'\gets B_{[k+1, n_r+k]}$
                \State $\bm{U}'\gets \bm{U}_{[k+1, n_r+k]}$
                \State $\bm{V}\gets\texttt{recReduce}(B', \bm{U}', \eta, \overline{\beta})$
                \State $\bm{V}'\gets \bm{E}_n;~\bm{B}\gets \bm{V}'\bm{B}$
                \State $\texttt{sizeReduce}(\bm{B})$
            \EndFor
        \EndFor
    \end{algorithmic}
\end{algorithm}
\mbox{}\\
\mbox{}\\
\mbox{}\\
\mbox{}\\
\end{minipage}
\hspace{0.03\columnwidth} % ここで隙間作成
\begin{minipage}[b]{0.45\columnwidth}
\begin{algorithm}[H]
    \footnotesize
    \begin{algorithmic}[1]
        \caption{\footnotesize $L^2$簡約\cite{Stehle10}}
        \label{alg_L2_2}
        \Require{パラメタ$\frac{1}{4}<\delta<1, \frac{1}{2}<\eta<\sqrt{\delta}$,基底$\{\bm{b}_1,\ldots,\bm{b}_n\}$}
        \Ensure{簡約された基底$\{\bm{b}_1,\ldots,\bm{b}_n\}$}
        \State \textbf{if}~$n=2$~\textbf{then}~Schonhageのアルゴリズム~\textbf{endif}
        \State $\bm{V}\gets \bm{E}_n$
        \For{$t=1$~to$(2n_r/n_{r-1})^2\lceil \log{(8\overline{\beta}/\eta)}\rceil$}
            \For{$k=2(n_r/n_{r-1})^2$~downto~$0$}
                \State $B'\gets B_{[kn_{r-1}/2+1, kn_{r-1}/2+n_{r-1}]}$
                \State $\bm{U}'\gets \bm{U}_{[kn_{r-1}/2+1, kn_{r-1}/2+n_{r-1}]}$
                \State $\bm{V}'=\texttt{recReduce}(B', \bm{U}', \eta, \overline{\beta})$
                \State $\bm{V}''\gets\bm{E}_{n_r}$
                \State $\bm{V}\gets \bm{V}''\bm{V}\bmod 2^{\overline{\beta}}$
                \State $\texttt{sizeReduce}(B, \bm{U})$
            \EndFor
        \EndFor
    \end{algorithmic}
\end{algorithm}
    \mbox{}\\
    \mbox{}\\
    \mbox{}\\
    \mbox{}\\
\end{minipage}
\end{frame}

\begin{frame}{Seysen簡約(1/3)}
\begin{itemize}
    \item $\{\bm{b}_1,\ldots, \bm{b}_n\}$: 基底
    \item $\bm{B}=(\bm{b}_1^\top, \ldots, \bm{b}_n^\top)^\top$: 基底行列
    \item $\bm{B}=\bm{RQ}~(\bm{R}:~\text{下三角行列},~\bm{Q}:~\text{直交行列})$
    \item $\bm{R}=\mqty[\bm{R}_{1, 1} & \bm{O}_{\lfloor n/2\rfloor, n-\lfloor n/2\rfloor}\\ \bm{R}_{2, 1} & \bm{R}_{2, 2}]$
\end{itemize}
\begin{definition}[Seysen簡約]
    $\{\bm{b}_1,\ldots, \bm{b}_n\}$がSeysen簡約されているとは
    \[
        n=1\lor \left(\bm{R}_{1, 1},~\bm{R}_{2, 2}\text{がSeysen簡約されている}\land \norm{\bm{R}_{2, 1}\bm{R}_{1, 1}^{-1}}_{\max}\le \frac{1}{2}\right)
    \]
\end{definition}
\end{frame}

\begin{frame}{Seysen簡約(2/3)}
    Seysen簡約は大まかには次のようなことを$\bm{B}=\bm{RQ}$なる$\bm{R}$に対して再帰的に行う.
    \begin{enumerate}
        \item $\bm{R}$が$1$次正方なら何もしない
        \item $\bm{R}$を$\mqty[\bm{R}_{1, 1} & \bm{O}\\ \bm{R}_{2, 1} & \bm{R}_{2, 2}]$とブロックに分ける
        \item $\bm{R}_{1, 1}, \bm{R}_{2, 2}$をそれぞれSeysen簡約
    \end{enumerate}
\end{frame}

\begin{frame}{Seysen簡約(3/3)}
\begin{algorithm}[H]
    \footnotesize
    \begin{algorithmic}[1]
        \caption{\footnotesize Seysen簡約\cite{DPS25}}
        \label{alg_seysen}
        \Require{下三角行列$\bm{R}\in M_{n}(\mathbb{R})$~s.t.~$\bm{B}=\bm{RQ}$}
        \Ensure{$\bm{UB}$が簡約基底行列となるようなunimodular行列$\bm{U}\in M_{n}(\mathbb{R})$}
        \If{$\bm{R}\in M_{1}(\mathbb{R})$}
            \State \textbf{return}~$[1]$
        \EndIf
        \State $\mqty[\bm{R}_{1, 1} & \bm{O}\\ \bm{R}_{2, 1} & \bm{R}_{2, 2}]\gets \bm{R}$ \Comment{$\bm{R}_{1, 1}\in M_{\lfloor n\rfloor}(\mathbb{R})$,$\bm{R}_{2, 1}\in M_{n-\lfloor n\rfloor, \lfloor n\rfloor}(\mathbb{R})$,$\bm{R}_{2, 2}\in M_{n-\lfloor n\rfloor, n-\lfloor n\rfloor}(\mathbb{R})$}
        \State $\bm{U}_{1, 1}\gets \texttt{seysenReduce}(\bm{R}_{1, 1});~\bm{U}_{2, 2}\gets \texttt{seysenReduce}(\bm{R}_{2, 2})$
        \State $\bm{R}_{2, 1}\gets \bm{U}_{2, 2}\bm{R}_{2, 1}$
        \State $\bm{U}_{2, 1}\gets \round{-\bm{R}_{2, 1}\bm{R}_{1, 1}^{-1}}$
        \State $\bm{R}_{2, 1}\gets \bm{U}_{2, 1}\bm{R}_{1, 1}+\bm{R}_{2, 1}$
        \State \textbf{return}~$\mqty[\bm{U}_{1, 1} & \bm{O}\\ \bm{U}_{2, 1}\bm{U}_{1, 1} & \bm{U}_{2, 2}]$
    \end{algorithmic}
\end{algorithm}
\end{frame}

\begin{frame}{BLASter LLL}{weakly-LLL簡約}
\begin{definition}[weakly-LLL簡約基底]
    $\{\bm{b}_1,\ldots,\bm{b}_n\}$がweakly-LLL簡約されているとは
    \[
        \norm{\bm{b}_k^\star}^2\ge (\delta - \mu_{k, k-1}^2)\norm{\bm{b}_{k-1}^\star}^2
    \]
    なるときをいう.
    \begin{itemize}
        \item[\quad]
        \begin{itemize}
            \item Lov\'asz条件のみを満たす
        \end{itemize}
    \end{itemize}
\end{definition}
\end{frame}

\begin{frame}%{\texttt{BLASter}~LLLアルゴリズム}
\begin{algorithm}[H]
    \footnotesize
    \begin{algorithmic}[1]
        \caption{\footnotesize BLASter LLLアルゴリズム\cite{DPS25}}
        \label{alg_blaster_lll}
        \Require{下三角行列$\bm{R}\in M_{n}(\mathbb{R})$~s.t.~$\bm{B}=\bm{RQ}$}
        \Ensure{$\bm{UB}$が簡約基底行列となるようなunimodular行列$\bm{U}\in M_{n}(\mathbb{R})$}
        \State $i_0\gets 0;~\bm{U}\gets \bm{E}_n$
        \Do
            \State $\bm{R}\gets \bm{B}=\bm{RQ}$なる下三角行列$\bm{R}~(\bm{Q}:~\text{直交行列})$
            \State $i_0\gets \ell/2-i_0$
            \State $\mathcal{I}\gets \{(i_0+k\ell+1, \min\{n, i_0+k\ell+\ell\})~|~0\le k<(n-i_0)/\ell\}$
            \State \textbf{for}~$(i, j)\in\mathcal{I}$~\textbf{do}~$\bm{V}_i\gets \texttt{LLL}(\bm{R}_{[i, j]\times [i, j]}, \delta)$~\textbf{endfor}\Comment{$\bm{V}_i$はLLL基底を与えるユニモジラ行列}
            \For{$(i, j)\in\mathcal{I}$}
                \State $\bm{B}_{[i, j]\times [1, n]}\gets \bm{V}_i\bm{B}_{[i, j]\times [1, n]};~\bm{U}_{[i, j]\times [1, n]}\gets \bm{V}_i\bm{U}_{[i, j]\times [1, n]}$
            \EndFor
            %\State \textbf{for}~$(i, j)\in\mathcal{I}$~\textbf{do}~$\bm{B}_{[i, j]\times [1, n]}\gets \bm{V}_i\bm{B}_{[i, j]\times [1, n]};~\bm{U}_{[i, j]\times [1, n]}\gets \bm{V}_i\bm{U}_{[i, j]\times [1, n]}$~\textbf{endfor}
            \State $\bm{R}\gets \bm{B}=\bm{RQ}$なる下三角行列$\bm{R}~(\bm{Q}:~\text{直交行列})$
            \State $\bm{W}\gets \texttt{seysenReduce}(\bm{R});~\bm{B}\gets\bm{WB};~\bm{U}\gets\bm{WU}$
            \State $f\gets \texttt{true}$\Comment{$\bm{B}$がweakly-LLL簡約されているか}
            \For{$i=1$to$n-1$}
                \State \textbf{if}~$\delta r_{i, i}^2> r_{i, i+1}^2+r_{i+1, i+1}^2$~\textbf{then}~$f\gets\texttt{false};~\textbf{break}$~\textbf{endif}
            \EndFor
        \doWhile{not~$f$}
    \end{algorithmic}
\end{algorithm}
\end{frame}

\begin{frame}[allowframebreaks]{参考文献}
\beamertemplatetextbibitems
\bibliographystyle{plain}
\typeout{}
\bibliography{ref}
\end{frame}

\end{document}
