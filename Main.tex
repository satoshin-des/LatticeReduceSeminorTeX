\RequirePackage{plautopatch}
\documentclass[12pt,aspectratio=169,xcolor=dvipsnames,table,dvipdfmx, leqno]{beamer}

\usepackage{bxdpx-beamer}
\usepackage{mathtools}
\usepackage{amsmath,amsthm,amssymb,amscd}
\usepackage{ascmac}
\usepackage{physics}
\usepackage{here}
\usepackage{url}
\usepackage{bm}
\usepackage{subcaption}
\usepackage{color}
\usepackage{multirow}
\usepackage[all]{xy}
\usepackage[T1]{fontenc}
\usepackage{newtxtext} % 数式以外をTXフォントで上書き
\usepackage{ulem}
\usepackage[deluxe,uplatex]{otf} % 日本語多ウェイト化
%\usepackage[linesnumbered,ruled]{algorithm2e}
\usepackage{algorithm}
\usepackage[noend]{algpseudocode}
\usepackage{tikz}


%Beamerの設定
\usetheme{Boadilla}

%Beamerフォント設定
\usefonttheme{professionalfonts} % Be professional!
\renewcommand{\familydefault}{\sfdefault}  % 英文をサンセリフ体に
\renewcommand{\kanjifamilydefault}{\gtdefault}  % 日本語をゴシック体に
\renewcommand{\Comment}[1]{\quad/*#1*/}
\newcommand{\vol}{\text{vol}}
\newcommand{\round}[1]{\left\lfloor #1 \right\rceil}
\newcommand{\bhline}{\noalign{\hrule height 1.0pt}}
\DeclareMathOperator{\Pot}{Pot}
\DeclareFontFamily{U}{mathx}{}
\DeclareFontShape{U}{mathx}{m}{n}{<-> mathx10}{}
\DeclareSymbolFont{mathx}{U}{mathx}{m}{n}
\DeclareMathAccent{\widecheck}{0}{mathx}{"71}

\algdef{SE}[DOWHILE]{Do}{doWhile}{\algorithmicdo}[1]{\algorithmicwhile\ #1}%

% Babel (日本語の場合のみ・英語の場合は不要)
\uselanguage{japanese}
\languagepath{japanese}
\deftranslation[to=japanese]{Theorem}{定理}
\deftranslation[to=japanese]{Lemma}{補題}
\deftranslation[to=japanese]{Example}{例}
\deftranslation[to=japanese]{Examples}{例}
\deftranslation[to=japanese]{Definition}{定義}
\deftranslation[to=japanese]{Definitions}{定義}
\deftranslation[to=japanese]{Problem}{問題}
\deftranslation[to=japanese]{Solution}{解}
\deftranslation[to=japanese]{Fact}{事実}
\deftranslation[to=japanese]{Proof}{証明}
\def\proofname{証明}

\setbeamertemplate{theorems}[numbered]
\newtheorem{Proposition}[theorem]{命題}

%タイトル
\title[勉強会]{勉強会形式ゼミ資料①\\P.Q.Nguyen and D. Stehle Floating-Point LLL Revisited}
\author[佐藤]{佐藤 新}
\date{\today}
%\institute[]{\inst{1}~立教大学}

\begin{document}
\maketitle

\begin{frame}
    本セミナーで用いられる記号など
    \begin{itemize}
        \item 全て$\{\bm{b}_1,\ldots,\bm{b}_n\}$を基底としてもつ整数格子
        \item $B=\max\{\norm{b}_i~|~1\le i\le n\}$
        \item 浮動小数点数の演算精度は$\ell$-bit
        \item $\diamond(a*b)$は$a*b$の浮動小数点演算$(*\in\{+, -, \times, /\})$
    \end{itemize}
\end{frame}

\begin{frame}{Gram-Schmidtの計算(1/5)}{今までのGSO情報の持ち方}
    Gram-Schmidtの情報は
    \[
    \mu_{i, j}=\frac{\langle \bm{b}_i, \bm{b}_j\rangle-\sum_{k=1}^{j-1}\mu_{j, k}\mu_{i, k}\norm{\bm{b}_k^\star}^2}{\norm{\bm{b}_j^\star}^2}, \norm{\bm{b}_i^\star}^2=\norm{\bm{b}_i}^2-\sum_{j=1}^{i-1}\mu_{i, j}^2\norm{\bm{b}_j^\star}^2
    \]
    という公式で計算可能

    \begin{itemize}
        \item 内積$\langle \bm{b}_i, \bm{b}_j\rangle$の計算に浮動小数点数(fpa)が必要\\
        \quad\quad $2^{-\ell}\norm{\bm{b}_i}\norm{\bm{b}_j}$の潜在的な不確定性がある
    \end{itemize}
\end{frame}

\begin{frame}{Gram-Schmidtの計算(2/5)}{$\text{L}^2$でのGSO情報の持ち方}
    Gram-Schmidtの情報から
    \[
    r_{i, j}=\langle \bm{b}_i, \bm{b}_j\rangle-\sum_{k=1}^{j-1}\mu_{j, k}r_{i, k},~\mu_{i, j}=\frac{r_{i, j}}{r_{j, j}}\quad(i\ge j)
    \]
    という公式で計算可能な形で情報を持つ\\
    $\quad\quad\rightsquigarrow$精度が改善される
    \begin{itemize}
        \item 内積はGram行列から得られる(fpa)である必要なし
        \item 乗算が\underline{\textbf{2回から1回}}に
    \end{itemize}
\end{frame}

\begin{frame}{Gram-Schmidtの計算(3/5)}{Lov\'asz条件の書き換え}
\[
s_j^{(i)}\coloneqq \norm{\bm{b}_i}^2-\sum_{k=1}^{j-1}\mu_{i, k}r_{i, k}\quad(1\le j\le i)
\]
とすると,Lov\'asz条件$\delta\norm{\bm{b}_{\kappa-1}^\star}^2\le \norm{\bm{b}_\kappa^\star}^2+\mu_{\kappa, \kappa-1}^2\norm{\bm{b}_{\kappa-1}^\star}^2$は
\begin{align*}
\norm{\bm{b}_\kappa^\star}^2+\mu_{\kappa, \kappa-1}^2\norm{\bm{b}_{\kappa-1}^\star}^2&=\norm{\bm{b}_\kappa}^2-\sum_{j=1}^{\kappa-2}\mu_{\kappa, j}^2\norm{\bm{b}_j^\star}^2=\norm{\bm{b}_\kappa}^2-\sum_{j=1}^{\kappa-2}\mu_{\kappa, j}r_{\kappa, j}\norm{\bm{b}_j^\star}^2\\
&=s_{\kappa-1}^{(\kappa)}
\end{align*}
より
\[
\delta r_{\kappa-1, \kappa-1}\le s_{\kappa-1}^{(\kappa)}
\]
と書き換えることができる.
\end{frame}

\begin{frame}{Gram-Schmidtの計算(4/5)}{基底の更新後のLov\'asz条件(1/2)}
更に$\bm{b}_{\kappa}$と$\bm{b}_{\kappa-1}$の交換後の基底
\[
\{\bm{c}_1,\ldots,\bm{c}_n\}\coloneqq\{\ldots,\bm{b}_{\kappa-2},\bm{b}_{\kappa}, \bm{b}_{\kappa-1}, \bm{b}_{\kappa+1},\ldots\}
\]
とすると,交換後に検証すべきLov\'asz条件は
\begin{equation}
    \label{lovasz_after_swap}
\delta\norm{\bm{c}_{\kappa-2}^\star}^2\le \norm{\bm{c}_{\kappa-1}^\star}^2+\nu_{\kappa-1, \kappa-2}^2\norm{\bm{c}_{\kappa-2}^\star}^2
\end{equation}
に移る.
\end{frame}

\begin{frame}{Gram-Schmidtの計算(5/5)}{基底の更新後のLov\'asz条件(2/2)}
$\norm{\bm{c}_{\kappa-2}^\star}^2=\norm{\bm{b}_{\kappa-2}^\star}^2,~\norm{\bm{c}_{\kappa-1}^\star}=\norm{\pi_{\kappa-1}(\bm{b}_\kappa)}^2$, $\nu_{\kappa-1, \kappa-2}=\mu_{\kappa-1, \kappa-2}$であるので,
\begin{align*}
    \text{式}\eqref{lovasz_after_swap}&\iff \delta\norm{\bm{b}_{\kappa-2}^\star}^2\le \norm{\pi_{\kappa-1}(\bm{b}_\kappa)}^2+\mu_{\kappa-1, \kappa-2}\norm{\bm{b}_{\kappa-2}^\star}^2\\
    &\iff \delta r_{\kappa-2, \kappa-2}\le \mu_{\kappa, \kappa-1}\norm{\bm{b}_{\kappa-1}^\star}^2+\norm{\bm{b}_\kappa^\star}^2+\mu_{\kappa-1, \kappa-2}\norm{\bm{b}_{\kappa-2}^\star}^2\\
    &\iff \delta r_{\kappa-2, \kappa-2}\le \norm{\bm{b}_\kappa}^2-\sum_{j=1}^{\kappa-3}\mu_{\kappa, j}r_{\kappa, j}\norm{\bm{b}_j^\star}^2=s_{\kappa-2}^{(\kappa)}
\end{align*}
になるので,$[s_j^{(i)}]$を持っておくと\textbf{追加のコストなく}次の条件に移れる
\end{frame}

\begin{frame}{}
\begin{algorithm}[H]
    \footnotesize
    \begin{algorithmic}[1]
        \caption{\footnotesize $L^2$内でのsize-reduction}
        \label{alg_size_L2}
        \Require{パラメタ$\eta>1/2$,$\kappa\in\mathbb{Z}$,基底$\{\bm{b}_1,\ldots, \bm{b}_n\}$}
        \Ensure{$[\bar{r}_{\kappa, j}],~[\bar{\mu}_{\kappa, j}],~[\bar{s}_j]~(j\le \kappa)$,$\{\ldots,\bm{b}_{\kappa-1},\bm{b}_{\kappa}',\bm{b}_{\kappa+1},\ldots\}$}
        \State $\bar{\eta}\gets\frac{\eta+1/2}{2}=\frac{2\eta+1}{4}$
        \Do
            \For{$j=1$ to $\kappa$}\Comment{Cholesky分解}
                \State $\bar{r}_{i, j}\gets \langle\bm{b}_i, \bm{b}_j\rangle$
                \State \textbf{for} $k=1$ to $j-1$ \textbf{do} $\bar{r}_{i, j}\gets \diamond(\bar{r}_{i, j}-\diamond(\bar{r}_{i, k}\bar{\mu}_{j, k}))$
                \State $\bar{\mu}_{i, j}\gets \diamond(\bar{r}_{i, j}/\bar{r}_{j, j})$
            \EndFor
            \State $s_0\gets \norm{\bm{b}_n}^2$
            \State \textbf{for} $k = 1$ to $j-1$ \textbf{do} $\bar{s}_j\gets \diamond(\bar{s}_{j-1}-\diamond(\bar{\mu}_{n, j}\bar{r}_{n, j}))$
            \State $r_{n, n}\gets s_n$
            \For{$i=\kappa-1$ downto 1}
                \State \textbf{if} $\abs{\bar{\mu}_{k, i}}\ge \bar{\eta}$ \textbf{then} $X_i\gets \round{\bar{\mu}_{k, i}}$ \textbf{else} $X_i\gets 0$
                \State \textbf{for} $j=1$ to $i-1$ \textbf{do} $\bar{\mu}_{k, j}\gets \diamond(\bar{\mu}_{\kappa, j}-\diamond(X_i\bar{\mu}_{i, j}))$
            \EndFor
            \State $\bm{b}_\kappa\gets \bm{b}_\kappa-\sum_{i=1}^{\kappa-1}X_i\bm{b}_i$
        \doWhile{$X\ne \bm{0}_{\kappa-1}$}
    \end{algorithmic}
\end{algorithm}
\end{frame}

\begin{frame}
\begin{algorithm}[H]
    \footnotesize
    \begin{algorithmic}[1]
        \caption{\footnotesize $L^2$簡約}
        \label{alg_L2}
        \Require{パラメタ$\eta, \delta$,基底$\{\bm{b}_1,\ldots,\bm{b}_n\}$}
        \Ensure{簡約された基底$\{\bm{b}_1,\ldots,\bm{b}_n\}$}
        \State $\bar{\delta}\gets(\delta+1)/2$
        \State $\bar{r}_{1,1}\gets\norm{\bm{b}_1}^2$
        \State $\kappa\gets 2$
        \While{$\kappa\le n$}
            \State size-reduce~(Algorithm~\ref{alg_size_L2})
            \State $\kappa'\gets\kappa$
            \While{$\kappa\ge 2 \land \bar{\delta}\bar{r}_{\kappa-1, \kappa-1}\ge \bar{s}_{\kappa-1}$}
                \State $\kappa\gets\kappa-1$
            \EndWhile
            \For{$i=1$ to $\kappa-1$}
                \State $\bar{\mu}_{\kappa, i}\gets\bar{\mu}_{\kappa', i};~\bar{r}_{\kappa, i}\gets\bar{r}_{\kappa', i};~\bar{r}_{\kappa, \kappa}\gets\bar{s}_{\kappa}$
            \EndFor
            \State $\{\bm{b}_1,\ldots,\bm{b}_n\}\gets\{\ldots, \bm{b}_{\kappa'-1}, \bm{b}_{\kappa'+1},\ldots,\bm{b}_{\kappa'}, \bm{b}_\kappa, \ldots\}$
            \State $\kappa\gets \kappa+1$
        \EndWhile
    \end{algorithmic}
\end{algorithm}
\end{frame}

\begin{frame}[allowframebreaks]{参考文献}
\beamertemplatetextbibitems
\bibliographystyle{plain}
\bibliography{ref}
\end{frame}
\end{document}